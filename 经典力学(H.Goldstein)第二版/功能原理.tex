\documentclass[a4paper]{article}
\usepackage{amsmath,amsfonts,amsthm,amssymb}
\usepackage{bm}
\usepackage{hyperref}
\usepackage{geometry}
\usepackage{yhmath}
\usepackage{pstricks-add}
\usepackage{framed,mdframed}
\usepackage{graphicx,color} 
\usepackage{mathrsfs,xcolor} 
\usepackage[all]{xy}
\usepackage{fancybox} 
\usepackage{xeCJK}
\newtheorem{theo}{定理}
\newtheorem*{exe}{题目}
\newtheorem*{rem}{评论}
\newmdtheoremenv{lemma}{引理}
\newmdtheoremenv{corollary}{推论}
\newmdtheoremenv{example}{例}
\newenvironment{theorem}
{\bigskip\begin{mdframed}\begin{theo}}
    {\end{theo}\end{mdframed}\bigskip}
\newenvironment{remark}
{\bigskip\begin{mdframed}\begin{rem}}
    {\end{rem}\end{mdframed}\bigskip}
\geometry{left=2.5cm,right=2.5cm,top=2.5cm,bottom=2.5cm}
\setCJKmainfont[BoldFont=SimHei]{SimSun}
\renewcommand{\today}{\number\year 年 \number\month 月 \number\day 日}
\newcommand{\D}{\displaystyle}\newcommand{\ri}{\Rightarrow}
\newcommand{\ds}{\displaystyle} \renewcommand{\ni}{\noindent}
\newcommand{\pa}{\partial} \newcommand{\Om}{\Omega}
\newcommand{\om}{\omega} \newcommand{\sik}{\sum_{i=1}^k}
\newcommand{\vov}{\Vert\omega\Vert} \newcommand{\Umy}{U_{\mu_i,y^i}}
\newcommand{\lamns}{\lambda_n^{^{\scriptstyle\sigma}}}
\newcommand{\chiomn}{\chi_{_{\Omega_n}}}
\newcommand{\ullim}{\underline{\lim}} \newcommand{\bsy}{\boldsymbol}
\newcommand{\mvb}{\mathversion{bold}} \newcommand{\la}{\lambda}
\newcommand{\La}{\Lambda} \newcommand{\va}{\varepsilon}
\newcommand{\be}{\beta} \newcommand{\al}{\alpha}
\newcommand{\dis}{\displaystyle} \newcommand{\R}{{\mathbb R}}
\newcommand{\N}{{\mathbb N}} \newcommand{\cF}{{\mathcal F}}
\newcommand{\gB}{{\mathfrak B}} \newcommand{\eps}{\epsilon}
\renewcommand\refname{参考文献}\renewcommand\figurename{图}
\usepackage[]{caption2} 
\renewcommand{\captionlabeldelim}{}
\begin{document}
\title{\huge{\bf{功能原理}}} \author{\small{叶卢
    庆\footnote{叶卢庆(1992---),男,杭州师范大学理学院数学与应用数学专业
      本科在读,E-mail:yeluqingmathematics@gmail.com}}}
\maketitle\ni
变动的外力$\mathbf{F}$对质点$P$作功,期间质点$P$从点$A$移动到点$B$.我们
质点,$\mathbf{F}$对质点所做的功为
$$
\int_C \mathbf{F}\cdot d\mathbf{s},
$$
其中$C$表示从$A$到$B$的质点所经过的路径,该路径是带有方向的,以$A$为起点,以
$B$为终点.则我们有
$$
\int_C \mathbf{F}\cdot d\mathbf{s}=m\int_{t_{1}}^{t_{2}}
\frac{d\mathbf{v}}{dt}\cdot
\mathbf{v}dt.
$$
其中$m$是质点的质量.根据分部积分公式,
$$
\int_a^bfG=F(b)G(b)-F(a)G(a)-\int_a^bFg,
$$
可得
$$
\int_{t_1}^{t_2}\frac{d\mathbf{v}}{dt}\cdot
\mathbf{v}dt=\mathbf{v}(t_2)^2-\mathbf{v}(t_1)^2-\int_{t_1}^{t_2}\mathbf{v}\cdot \frac{d\mathbf{v}}{dt}dt.
$$
于是可得
$$
\int_{t_1}^{t_2}\frac{d\mathbf{v}}{dt}\cdot \mathbf{v}dt=\frac{1}{2}(\mathbf{v}(t_2)^2-\mathbf{v}(t_1)^2),
$$
于是,
$$
\int_C\mathbf{F}\cdot d\mathbf{s}=\frac{1}{2}m(\mathbf{v}(t_2)^2-\mathbf{v}(t_1)^2).
$$
这就是功能原理.
\begin{remark}
也可以这样推导:
\begin{align*}
  \int_C \mathbf{F}\cdot d\mathbf{s}=m\int_{t_{1}}^{t_{2}}
\frac{d\mathbf{v}}{dt}\cdot
\mathbf{v}dt&=m\int_{v_1}^{v_2}\mathbf{v}\cdot d\mathbf{v}\\&=m(\frac{\mathbf{v}(t_{2})^2}{2}-\frac{\mathbf{v}(t_{1})^2}{2}).
\end{align*}
\end{remark}
\end{document}
